\documentclass[11pt]{article}
\usepackage{polski}
\usepackage[utf8]{inputenc}
\usepackage{mathtools}
\usepackage{amsmath}
\usepackage[T1]{fontenc}
\usepackage{times}
\usepackage{courier}
\usepackage{enumerate}



\begin{document}

\begin{center}\large{ \textbf{PROGRAMOWANIE LINIOWE - ZADANIA}} \end{center}

\textbf{Zadanie 1}\\
Przedsiębiotrstwo produkuje dwa produkty $P_1$ i $P_2$. W procesie ich wytworzenia wykorzystuje trzy surowce: $S_1$, $S_2$ i $S_3$.
Zużycie poszczególnych surowców do produkcji jednostki każdego z produktów zebrano w tabeli. W ostatniej kolumnie tabeli przedstawiono limity poszczególnych surowców dostępnych w przedsiębiorstwie dla miesięcznej produkcji.


\begin{center}
%\begin{table}
\begin{tabular}{|c||c|c||c|}

\hline
surowce  produkty&$P_1$&$P_2$&limit\\
\hline
\hline
$S_1$&3&2&7\\
$S_2$&3&4&8\\
$S_3$&1&5&8\\
\hline
\end{tabular}\\
%\caption {moja tabela}
%\end{table}
\end{center}
Przyjmując, że produkty $P_1$ i $P_2$ można sprzedać odpowiednio po cenach $c_1=3$ PLN i $c_2=5$ PLN, określ strukturę produkcji, która pozwalała by uzyskać maksymalny przychód.\\
W tym celu wykonaj następujące polecenia:
\begin{enumerate}[a)]
\item Zapisz zadanie maksymalizacji (będące zadaniem prymalnym) w postaci standardowej, tzn. określ funkcję celu $Z$ i podaj warunki ograniczające definiujące zbiór rozwiązań dopuszczalnych (ZRD).
\item Podaj wymiar zadania, tzn. liczbę zmiennych decyzyjnych $n$, liczbę funkcyjnych warunków ograniczających $m$ oraz łączną liczbę warunków ograniczających.
\item Czy zadanie jest przykładem programowania liniowego?
\item Czy zadanie prymalne można rozwiązać przy pomocy metody geometrycznej czy tylko przy pomocy metody simpleks?
\item Podaj górne ograniczenie liczby iteracji dla metody geometrycznej i metody simpleks? Czy te liczby są sobie równe?
\item Jeżeli rozwiązujesz zadanie metodą simpleks, zapisz je w postaci kanonicznej.
\item Wyznacz optymalne rozwiązanie, maksymalizujące funkcję celu: $\mathbf{x^\ast}=\begin{bmatrix}x_1^\ast & x_2^\ast \end{bmatrix} '$.
\item Wyznacz wartość funkcji celu odpowiadającą rozwiązaniu optymalnemu: $Z^\ast$.
\item Które funkcyjne warunki ograniczające  dla rozwiązania optymalnego są spełnione w formie równości, a które w formie ostrej nierówności.
\item Określ ceny dualne surowców $S_1$, $S_2$ i $S_3$. Jeżeli stosujesz metodę geometryczną do określenia cen dualnych wykorzystaj rozwiązanie problemu dualnego z \textbf{Zadania 2}. Zinterpretuj znaczenie cen dualnych surowców.
\item$^\ast$ Wykonaj analizę wrażliwości w wariancie analizy przedziałowej parametrów funkcji celu (poszczególne ceny sprzedaży produktów) oraz wyrazów wolnych funkcyjnych warunków ograniczających (zasoby poszczególnych surowców).   
\end{enumerate}

\textbf{Zadanie 2}\\
Dla zagadnienia maksymalizacji przedstwaionego w \textbf{Zadaniu 1} rozważ zadanie dualne oraz wykonaj następujące polecenia: 
\begin{enumerate}[a)]
\item Zapisz zadanie dualne względem pierwotnego zadania maksymalizacji w postaci standardowej, tzn. zapisz jego funkcję celu $K$, określ kierunek jej optymalizacji (minimalizacja czy maksymalizacja) oraz podaj warunki ograniczające dla tego zadania, tworzące definicję ZRD. 
\item Podaj wymiar zadania dualnego, tzn. liczbę dualnych zmiennych decyzyjnych, liczbę funkcyjnych warunków ograniczających oraz łączną liczbę warunków ograniczających. Czy istnieją w tym względzie zależności pomiędzy zadaniem prymalnym a dualnym.
\item Rozwiąż zadanie metodą geometryczną wykorzystując uzyskane rozwiązanie optymalne zadania prymalnego oraz twierdzenia dotyczące zależności pomiędzy optymalnym rozwiązaniem zadania dualnego i prymalnego.
\item Jeżeli rozwiązałeś zadanie prymalne metodą simpleks, zastanów się czy rozwiązanie optymalne dla zadania dualnego z tablicy simpleks dla rozwiązania optymalnego zadania prymalnego.
\item Czy istnieje związek pomiędzy wartością funkcji celu zadania prymalnegodla rozwiązania optymalnego $Z^\ast$, a wartością optymalną $K^\ast$ funkcji celu zadania dualnego. 
\end{enumerate}


\end{document}
